\documentclass{scrartcl}


\usepackage{texenv}
\begin{document}

%title
%\noindent\makebox[\linewidth]{\rule{\textwidth}{0.2pt}}
{\Large \centering \textsf{IT-Systeme: "Ubungsblatt 07} -- Florian Macherey, 
\today}\\
\noindent\makebox[\linewidth]{\rule{\textwidth}{0.2pt}} \\

%\normalsize \flushleft

\section*{Aufgabe 1: Bufferoverflows}
\subsection*{1a) Beschreiben Sie, was bei einem Bufferoverflow passiert?}
\subsection*{1b) Wo liegt die Gefahr von Bufferoverflows?} 
\subsection*{1c) Wie kann ein Bufferoverflow verhindert werden?}

\section*{Aufgabe 2: XSS \\Vergleiche sie die Cross-Site-Scripting-Attacke mit einer SQL-Injection.}
\subsection*{2a) Welche Gemeinsamkeiten gibt es?}
\subsection*{2b) Welche Unterschiede gibt es?}

\section*{Aufgabe 3: Password cracking }
Quellcode zum Password cracking. Es wird ein Hybridverfahren verwendet, bei welchem zuerst h"aufig verwendete Passw"orter aus einen W"ortbuch ausprobiert werden, danach werden alle "ubrigen Kombinationen per Brute-Force ermittelt. Wenn ein Passwort bzw. sein Hashwert erfolgreich gefunden wurde, wird das Programm beendet, ohne die weitern M"oglichkeiten zu testen.
%todo quellcode einfuegen 
\lstinputlisting[language=Java,]{ue07.03/Encrypt.java}

\section*{Aufgabe 4: Bonusaufgabe \\Legen sie das Wort \dq Password\dq ~aus Crackern und machen Sie ein Foto davon.}
%todo Bild einfuegen
\end{document}
