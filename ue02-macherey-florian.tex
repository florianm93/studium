\documentclass{article}

\usepackage{texenv}

\begin{document}

%title
%\noindent\makebox[\linewidth]{\rule{\textwidth}{0.2pt}}
{\Large \centering \textsf{IT-Systeme: "Ubungsblatt 02} -- Florian Macherey, \today}\\
\noindent\makebox[\linewidth]{\rule{\textwidth}{0.2pt}} \\

%\normalsize \flushleft

\section*{Aufgabe 1: CPU}
\subsection*{1.a) Was ist ein Mikroprogramm?}
Ein Mikroprogramm ist der Kern einer CISC (Complex Instruction Set Computer) 
CPU. Mit diesem werden wenige, daf"ur aber sehr komplexe Befehle in der CPU 
codiert. Beispielswiese wird die Division damit implementiert. Au\ss erdem sind 
die Ausf"uhrungszeiten der Befehle unterschiedlich lang. 

\subsection*{1.b) Was ist Pipelining? Welche Probleme sehen Sie hier beim 
Multitasking?}
Beim Pipeling werden aus Laufzeitgr"unden die Befehle nicht seriell hintereinander 
sondern parallel ausgef"uhrt. Dabei muss beachtet werden das sie nicht echt 
parallel ausgef"uhrt werden, sondern ein Prozess nur eine Resource gleichzeitig 
belegen kann. \\

Das Problem bei der parallelen Ausf"uhrung ist, dass es Dateninkonsitenten geben 
kann, wenn ein Prozess auf die Daten eines anderen Prozesses zugreift, welche 
durch diesen ver"andert werden. 

\section*{Aufgabe 2: Speicher}
\subsection*{2.a) Erkl"aren Sie die \dq Speicherpyramide\dq ~und ihre 
Bedeitung f"ur den Aufbau eines Computers.}
Die Speicherpyramide beschriebt den Zusammenhang von von Kosten $/$ Bit und 
Kapazit"at, Zykluszeit und Zugriffszeit.  Je h"oher eine Art Speicher in der 
Pyramide steht, desto gr"o\ss er sind die Kosten f"ur ein Speichermodul. 
Gleichzeitig sind die Kapazit"at, Zykluszeit und Zugriffszeit kleiner bzw. geringer. 
\\

Von oben nach unten angeordnet sind in der obersten Stufe die Register bzw. der 
On Chip Cache, in der zweiten Stufe der Cache (SRAM), in der dritten Stufe der 
EDO RAM, FPM RAM, SDRAM und Rambus DRAM, in der vierten Stufe 
Peripherieger"ate und in der f"unften und letzten Stufe die externen Speicher.   

\subsection*{2.b) In welchen Einheiten werden Speicherinhalte 
"ublicherweise geladen? Wo finden Sie diese Einheiten wieder?}
Ein Cachespeicher enth"alt ein Array von Sets, in diesen Sets werden eine oder 
mehrere Zeilen gespeichert. In jeder Zeile wird ein Block gespeichert. Diese 
Variante der Speichereinheiten werden auch bei den Massenspeichern eingesetzt.  

\section*{Aufgabe 3: Cache}
\subsection*{3.a) Erkl"aren  Sie das Prinzip der \dq "ortlichen Lokalit"at\dq 
~und warum es f"ur die Verwendung von Caches wichtig ist. Geben Sie auch 
ein Beispiel, um ihre Erkl"arung zu veranschaulichen.}
Die "ortliche Lokalit"at beschriebt, dass wenn entsprechend programmiert wird, 
nachdem die Speicherzelle $A$ gelesen wurde, auch die Speicherstelle $B=A+k$ 
f"ur $k$ klein gelesen wird. Es ist bei der Verwendung sehr wichtig, damit die 
Caches effizient arbeiten k"onnen, da sie sehr kurze Antwortzeiten realisieren 
k"onnen. \\

Ein Beispiel f"ur eine gute "ortliche Lokalit"at ist zum Beispiel, wenn eine Addition 
von zwei Zahlen durchgef"uhrt werden soll, dann ist praktisch wenn die zwei 
Zahlen in zwei Pages direkt nebeneinander oder nah beieinander im Cache 
liegen.  

\subsection*{3.b) Erkl"aren Sie die interne Funktionsweise eines \dq Fully 
Associate Cache\dq . Geben Sie auch ein Beispiel um Ihre Erkl"arung zu 
veranschaulichen.}
Bei einem Fully Associate Cache kann der Block an jeder Stelle im Cache eingef"ugt 
werden. Die Addresse wird mithilfe einer Modulo-Rechnung berechnet und in dem 
entsprechenden Block gespeichert. Au\ss erdem wird diese Adresse in den Tag 
des Speicherblocks eingetragen. \\

Wenn zum Beispiel 8 Cachepl"atze verf"ugbar sind, und Block 11 eingetragen 
werden soll, wird dieser an der Stelle $11\,\, mod\,\, 8 = 3$ eingetragen. 
Weiterhin 
wird wie oben erw"ahnt dann die 3 in den Tag des Blocks eingetragen. 
\end{document}
