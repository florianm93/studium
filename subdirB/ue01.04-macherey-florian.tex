\documentclass{article}

\usepackage{german}
\usepackage[utf8]{inputenc}
\usepackage[a4paper, top=30mm, left=25mm, right=25mm, bottom=25mm]{geometry}
\usepackage{booktabs}
\usepackage[usenames,dvipsnames]{xcolor}
\usepackage{listings}
\usepackage{MnSymbol}

\parindent=0pt

% own color definitions:
\definecolor{TeaGreen}{HTML}{FFFFBB}

% setting for lstlistings
\lstset{
	basicstyle=\small,
	breakatwhitespace=true,
	breaklines=true,
	commentstyle=\color{ProcessBlue},
	keepspaces=true,
	keywordstyle=\color{red},
	rulecolor=\color{black}, 
	numbers=left,
	stringstyle=\color{RawSienna},
	firstnumber=1,
	numberfirstline=true,
	stepnumber=5,
	tabsize=4,
	language=[x86masm]Assembler,
    backgroundcolor=\color{TeaGreen},
	prebreak=\raisebox{0ex}[0ex][0ex]{\ensuremath{\swarrow}},
	postbreak=\raisebox{0ex}[0ex][0ex]{\ensuremath{\rcurvearrowse\space}},	
}


\begin{document}

%title
{\Large \centering \textsf{IT-Systeme: "Ubungsblatt 01} -- Florian Macherey, \today}\\
\noindent\makebox[\linewidth]{\rule{\textwidth}{0.2pt}} \\

\section*{Aufgabe 4 \textit{Was macht das folgende Programm?}}
\begin{lstlisting}
mov esp,0x50
jmp %start

%attack: 
mov [esp], 1
ret
           
%sub:
mov eax, ebx
add eax, ecx
ret
                        
%start:
mov eax,0
mov ebx, 5
mov ecx, 7
call %sub
                                                
mov eax, 0
mov ebx, 12
mov ecx, 11
call %sub
                                                                    
call %attack
\end{lstlisting}
Das Programm l"auft in einer Endlosschleife. Es wird zuerst die Summe von 5 und 7, danach die von 12 und 11 berechnet. Dies wird jeweils in der Untermethode \texttt{sub} ausgef"uhrt. Danach wird in der \texttt{attack}-Methode ein Wert in dem Stack gesetzt. 

\end{document}
