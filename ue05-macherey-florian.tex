\documentclass{article}


\usepackage{texenv}
\usepackage{multirow}

\begin{document}

%title
%\noindent\makebox[\linewidth]{\rule{\textwidth}{0.2pt}}
{\Large \centering \textsf{IT-Systeme: "Ubungsblatt 05} -- Florian Macherey, 
\today}\\
\noindent\makebox[\linewidth]{\rule{\textwidth}{0.2pt}} \\

%\normalsize \flushleft

\section*{Aufgabe 1: Rechnerarchitekturen}
\subsection*{1a) Was sagt \dq Amdahl's Law\dq ~in Bezug auf die 
Geschwindigkeitszunahme von parallelen Programmen auf geeigneten 
Rechnerarchtikturen aus?}
Ber der Berchnung wie stark ein Programm parallelisierbar ist, ber"ucksichtigt das  
\dq Amdahl's Law\dq ~au\ss erdem den seriellen, also nicht parallelisierbaren 
Teil. Die Formel f"ur die Berechnung ist damit:
\[
 S(p) = \dfrac{T(1)}{T(p)} = \dfrac{T(1)}{f*T(1) + (1-f)*\dfrac{T(1)}{p} = 
 \dfrac{1}{f + 
 \dfrac{1-f}{p}} }
\]
wobei \textit{f} der serielle Anteil ist, \textit{S(p)} der Speedup, \textit{T(1)} und 
\textit{T(p)} der Zeitverbrauch mit einer beziewhungsweise \textit{p} CPUs und 
\textit{p} die Anzahl der CPUs sind.  

\subsection*{1b) Sind die Aussagen von \dq Amdahl's Law\dq ~realistisch? 
Wenn nein, warum nicht? Geben Sie ein Beispiel}
Die Aussagen von \dq Amdahl's Law\dq ~sind nicht realistisch, da zum Beispiel 
der Overhead nicht ber"ucksichtigt wird. Au\ss erdem kann durch zum Beispiel 
Caching Verfahren der ideale Speedup gr"o\ss er werden, in der Regel ist der 
Speedup aber geringer als beim \dq Amdahl's Law\dq.

\section*{Aufgabe 2: Multithreading}
\subsection*{2a) Erkl"aren Sie den Begriff \dq Deadlock\dq. Geben Sie auch 
ein Beispiel um Ihre Erkl"arung zu veranschaulichen.}
Der Deadlock ist der Zustand wenn zwei oder mehr Prozesse sich gegenseitig 
blockieren, weil sie auf Resourcen warten, die der jeweils andere Prozess noch 
gesperrt hat. Da sie beide die andere Resource ben"otigen um weiter zu machen, 
k"onnen sie ihre aktuelle Resourcen nicht freigeben. \\
Als L"osungsstrategie kann man zum Beispiel zum Beginn der Laufzeit alle n"otigen 
Locks anfordern, dies ist aber im allgemeinen ineffizient. Deshalb k"onnen die 
vorhandenen Locks zuf"allig freigegeben werden, damit andere Prozesse diese 
Locks anfordern k"onnen. Auch eine M"oglichkeit ist, wenn nach einer fest 
definierten Zeit nicht alle Locks angefordert werden konnten, alle vorhandenen 
Resourcen freizugeben, damit blockierte Prozesse diese locken k"onnen. 

\subsection*{2b) Erkl"arung zu den Threads und dem Programm}
In dem Programm wird die Berechnung des Matrixproduktes in einzelne Threads 
zerlegt. Jeder Thread berechnet dabei einen Teil der Matrix, aber mindestens die 
Zeilen. W"urden die Threads nur einzelne Matrixelemente berechnen, w"are der 
Overhead daf"ur zu gro\ss. Deswegen gibt es auch eine Abfrage, dass es maximal 
soviele Threads wie Dimensionen der Matrix gibt.

\end{document}
